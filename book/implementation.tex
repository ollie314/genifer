\chapter{Implementation}

\begin{flushright}
\emph{The sooner you start coding your program the longer it's going to take}\\
--- H F Ledgard 1975
\end{flushright}

\begin{flushright}
\emph{Premature optimization is the root of all evil.}\\
--- Donald Knuth
\end{flushright}

\begin{flushright}
\emph{The First Rule of Program Optimization: Don't do it.\\
The Second Rule of Program Optimization (for experts only!): Don't do it yet.}\\
--- Michael A Jackson
\end{flushright}

\minitoc

\section{Choice of programming language}

\textbf{Lisp} is the single most important programming language ever invented in the history of computer science.  Much of the research code in classical AI was written in Lisp, and to this day Lisp (and dialects like Scheme) remains a very practical language (I use it for rapid prototyping).  ``Necessary but not sufficient'':  If you don't know Lisp you're not an AI expert, but knowing or using Lisp doesn't necessarily mean you'll succeed in AGI.

Lisp is also based on $\lambda$-calculus which is important in the study of logic.

\textbf{Prolog} seems not as good as Lisp because it imposes the restriction of Horn clauses (over full first-order logic), and SLD resolution with a depth-first search strategy.  (Though Peter Norvig pointed out this is not a severe limitation of Prolog as compared to Lisp.)

Interestingly, a logic-based AGI \textit{itself} is like an advanced-version Prolog interpreter, enhanced with better search strategies (eg best-first search), probabilities / fuzziness (eg fuzzy Prolog), higher-order unification (as in $\lambda$-Prolog), abduction (as abductive logic programming), induction (as inductive logic programming), etc.  Thus, a good understanding of Prolog is essential to the study of AGI.

\textbf{ML} was created by Robin Milner for the purpose of automated theorem proving.  ML's type system helps to ensure that theorems are proved correctly.  ML is used to develop the LCF (logic of computable functions) series of theorem provers, which influenced HOL, Isabelle, and HOL Light.  OCaml is derived from ML.  The "CAM" in OCaml stands for ``categorical abstract machine'' which is based on categorical combinatory logic, a variant of combinatory logic combined with category theory.

\textbf{Haskell} is the programming language closest to mathematics, which makes it very elegant.  The optimizing compiler built by Simon Peyton Jones at Glasgow makes it a very fast language in recent benchmarks.  Lazy evaluation is also a strong point when implementing symbolic AI algorithms.  So I think Haskell is a strong contender to Lisp as the most suitable AGI language.

\subsubsection{Low-level languages}

Object-orientation is not particularly natural for some software architectures.

\textbf{Java} is preferable to C\# for its cross-platform maturity.

\textbf{C} may be too old.  (But personally I prefer C to C++.)

\textbf{C++?}  Not bad for AGI, in my opinion.  C++ is also the choice of OpenCog.
