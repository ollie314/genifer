\chapter{Memory organization}
\begin{flushright}
\emph{Computer science has only three ideas: cache, hash, trash}\\ --- Greg Ganger, CMU
\end{flushright}
\minitoc

\section{Associative memory}
\label{sec:associative-memory}

Associative recall can greatly facilitate inductive learning.

Sometimes fuzzy associations may be desirable.

The KB being in first-order logic poses problems for fuzzy associations.  It is unlike associative neural networks which may be closer to propositional logic.

\section{Efficient rule selection}

Given:\\
1. The goal (ie, the conclusion of the proof)\\
2. The premises (ie, facts residing in Working Memory)\\
Try to predict:\\
3. The rules involved in the proof

Each data point would be:\\
1.  goal (a grounded fact)\\
2.  premises (set of grounded facts)\\
3.  a rule used in the derivation of the goal

So the function we need is:\\
\{ fact \} $\times$ \{ set of facts \} $\rightarrow$ rule

Using multidimensional scaling, we can map a fact to its coordinates in a high-dimensional space.

Then we can partition the high-dimensional space into grids, and to each grid assign a bucket of rules.  We need some way to partition the high-dimensional space.  But we can also partition the space of data points into many categories?

\subsection{Hierarchical clustering of oracles}
\label{sec:hi-oracle}

\underconst