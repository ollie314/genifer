\chapter{Belief revision}
\label{ch:belief-revision}

\minitoc

Belief revision concerns the problem of conflicting conclusions and, in the extreme, contradictions.

One simple question is:  If we arrive at two P(Z) distributions from two separate inference chains, how to combine the answers?

It seems that this is related to statistical inference.  Suppose we have a population.  A set of observations gives the estimate of the mean and variance:\\
\hspace*{1cm} $x_1, x_2, ..., x_{n_1} \sim Beta(\mu_1, v_1)$\\
Another set of observations gives another pair of estimates:\\
\hspace*{1cm} $y_1, y_2, ..., y_{n_2} \sim Beta(\mu_2, v_2)$\\
The question is to find a reasonable way to combine the 2 sets of observations to give a new estimate:\\
\hspace*{1cm} $x_1, x_2, ..., y_1, y_2, ... \sim Beta(\mu_0, v_0)$\\
In other words, how to express $(\mu_0, v_0)$ in terms of $(\mu_1, v_1)$ and $(\mu_2, v_2)$?

Maybe the $\mu_0$ is just the weighted average of $\mu_1$ and $\mu_2$?  $n_1, n_2$ can be assumed to be equal to the supports $w_1, w_2$ which can be calculated from the confidences $c_1, c_2$.

Some assumptions in the above may not be entirely justified:  the population size is typically small and the 2 samples may have overlap, ie, not independent.

\section{Justifications}

Justifications are the reasons why we believe in something.

\textbf{Justification-based Truth Maintenance Systems} (JTMS) keep track of justifications explicitly.  They are needed because otherwise we would not be able to recall why certain beliefs are in the mind.

\{ TO-DO:  For example... \}

This may be a requirement in addition to probabilistic reasoning.

\section{Many-worlds representation}

\begin{flushright}
\emph{The test of a first-rate intelligence is the ability\\
to hold two opposed ideas in the mind at the same time,\\
and still retain the ability to function.}\\
--- F Scott Fitzgerald
\end{flushright}

With probabilistic logic, it may be possible to represent multiple possible worlds in a single KB, by storing multiple potentially-conflicting assumptions in the KB.

For example, in the \textit{Truman Show} movie, there may be 2 competing assumptions in Truman's mind:\\
\hspace*{1cm} A1.  Truman's world is normal.\\
\hspace*{1cm} A2.  Turman's world is populated by fake actors.

In binary logic we can only accept either A1 or A2, thus the belief revision problem becomes unnecessarily much harder.

If we use probabilistic logic, then all we need is to impose that probabilities of alternative assumptions (ie, mutually exclusive or disjoint events) sum to 1.  Disjoint means that $P(A \cap B) = 0$, ie, A and B cannot be true at the same time.  The general rule is:\\
\hspace*{1cm} If we have a set of assumptions such that $P(A_i \cap A_j) = 0 \; \mbox{for all} \; i \neq j$,\\
\hspace*{1cm} then $\sum P(A_i) = 1$.

\subsection{Deliberative assumptions and ATMS}

Sometimes we make deliberative assumptions for more efficient reasoning, for example:  \textit{``Assuming the teacher will not check, I copy my classmate's homework''}.

This is in contrast to the kind of \textbf{implicit assumptions} described above, which are treated probabilistically, for example:  ``The teacher may check my homework with probability 0.2''.

The difference is that we deliberately force the reasoner to speculate on the consequences of an assumption as if its probability is 1.

This kind of reasoning is similar to binary logic, in which we must be careful about keeping track of conflicting assumptions.  \textbf{Assumption-based Truth Maintenance Systems} (ATMS) are developed specifically for this purpose.

ATMS seems somewhat redundant if we have probabilistic logic, where we can deal with multiple assumptions without probabilistic conflict (which in binary logic would result in conflicts).

If an assumption is particularly important, a probabilistic reasoner can take expected utilities into consideration.

\section{Consistency check}
\label{sec:consistency-check}

A few types of consistency checks are needed.

\section{Conflict resolution}
\label{sec:conflict-resolution}

Ie, what to do when an inconsistency is found.

\section{Theory revision}

See \S\ref{sec:inductive-learner}.

