\chapter*{A. Quick start guide to AGI}
\addcontentsline{toc}{chapter}{Appendix A: Quick start guide to AGI}

This is the quickest way to get up to speed from 0 to AGI.  (Warning: These recommendations are subjective!)

If you want to get a quick understanding of \textbf{neuroscience} (strictly speaking you may not even need to):\\
Browse, but don't read:
\begin{compactenum-}[\textbullet ]
\item a book about the human brain, such as: \href{http://www.amazon.com/Human-Brain-Introduction-Functional-Anatomy/dp/0323041310/ref=sr_1_2?ie=UTF8&s=books&qid=1268965281&sr=8-2}
{The Human Brain}
\item a textbook of neurochemistry, such as:
\href{http://www.amazon.com/Basic-Neurochemistry-Seventh-Molecular-Cellular/dp/012088397X/ref=sr_1_1?ie=UTF8&s=books&qid=1268965399&sr=1-1}
{Basic Neurochemistry}
\item a textbook on the neuron, such as:
\href{http://www.amazon.com/Neuron-Cell-Molecular-Biology/dp/0195145232/ref=sr_1_1?ie=UTF8&s=books&qid=1268965470&sr=1-1}
{The Neuron}
\item a book on modeling a single neuron, such as:
\href{http://www.amazon.com/Biophysics-Computation-Information-Computational-Neuroscience/dp/0195181999/ref=sr_1_1?ie=UTF8&s=books&qid=1268967514&sr=1-1}
{Biophysics of Computation: Information Processing in Single Neurons}
\item a book on modeling the whole brain, such as:
\href{http://www.amazon.com/Memory-Attention-Decision-Making-computational-neuroscience/dp/0199232709/ref=ntt_at_ep_dpt_2}
{Memory, Attention, and Decision-Making: A unifying computational neuroscience approach}
\end{compactenum-}
Then you will get an idea of the complexity of wetware and a general sense of how intractable it is to re-engineer the brain (except by brute force).  So we should give it up.  Note the analogy:  it is easier to engineer a flying machine with a novel design rather than exactly copying the bird.

\underconst

To learn the basics of AI (the status quo of current AI), get one of these AI bibles:
\begin{compactenum-}[\textbullet ]
\item AIMA
\item Luger
\item Winston
\item Nilsson
\end{compactenum-}

To learn logic:
\begin{compactenum-}[\textbullet ]
\item Chang $\&$ Lee
\item Fitting
\item higher-order logic:
\item lambda calculus: ?
\item combinatory logic: ?
\item term rewriting: ?
\end{compactenum-}

The best books to learn programming languages (in the context of AI):
\begin{compactenum-}[\textbullet ]
\item Prolog: Ivan Bratko
\item Lisp: PAIP or Winston
\item ML: Paulson: ML for the Working Programmer
\item Haskell: ``Algorithms'' by Rabhi $\&$ Lapalme
\end{compactenum-}

The best books on:
\begin{compactenum-}[\textbullet ]
\item Inductive logic programming: de Raedt
\item Causality: Williamson
\item Fuzzy logic: ?
\item Bayesian networks: Pearl
\item Knowledge representation: Levesque $\&$ Brachman
\item Natural language understanding: Jurafsky $\&$ Martin
\end{compactenum-}
